\documentclass[CJK]{beamer}
\mode<presentation>{}

\usepackage{url}

\usepackage{luatexja-fontspec}
\setmainjfont{KaitiSC}
\usefonttheme[onlymath]{serif}
\setbeamercolor{math text}{fg=black!15!blue}
\linespread{1.3}
\usepackage{tikz} 
\usetikzlibrary{intersections,backgrounds}

\usepackage{systeme}

%% preamble
\title{线性代数入门}
\subtitle{几个基本概念}
\author{Zhihu: tempo}
\date{2016 年 11 月 6 日}

\graphicspath {{images/}}

\begin{document}

%% title frame
\begin{frame}
\titlepage
\end{frame}

\begin{frame}
\frametitle{写在前面}
\begin{itemize}
	\item 提纲里已经说了,这次 Live 涉及的是比较基本的概念。有些概念可能部分人已经学过,但还是希望大家耐心些。末尾的互动环节可以自由提问。
	\item 关于方法论简单讲几句:
	\begin{enumerate}
	\setlength\itemsep{0.5em}
	\item 重视 ``算法''(参见前面的讨论)。
	\item 对初学者而言,算法只有一个,即\alert{高斯消元法},待会会详细讲。
	\item 重视概念:\alert{线性映射}的概念最重要,然后是抽象的\alert{线性空间}。
	\item 熟悉代数运算
    \end{enumerate}

\end{itemize}
\end{frame}

\begin{frame}
\frametitle{写在前面}
\begin{itemize}
	\setlength\itemsep{1.5em}
	\item 由于 Live 的限制,我会尽量连贯地用语音说,说 15-20 分钟再停下来搜集一下反馈,否则信息的强度没法保证。
    \item 剩下的时间主要讨论数学。

\end{itemize}
\end{frame}

%% normal frame
\begin{frame}
\frametitle{开讲}
考虑一个简单的问题,如何描述:
\[
\systeme*{w + x + y - z = 0 , x - 2y + 3z = 0}
\]
的解集?($w, x, y, z\in \mathbb{R}$)
\end{frame}

\begin{frame}
\frametitle{中学就会的技巧}
	第一个方程减去第二个方程,得
\[
\systeme*{w + 3y - 4z = 0, x - 2y + 3z = 0}
\]
移项
\[
\systeme*{w = -3y + 4z, x = 2y - 3z}
\]
\end{frame}

\begin{frame}
\frametitle{观察}
对任意取定的$y = y_0, z = z_0\in \mathbb R$, 由
\[
\systeme*{w = -3y + 4z, x = 2y - 3z}
\]
确定 $w, x$ 的值,则得到唯一的一组解 $(w,x,y,z)$.
\[
\systeme*{w = -3y_0 + 4z_0, x = 2y_0 - 3z_0, y = y_0, z = z_0}
\]
\end{frame}

\begin{frame}
\frametitle{什么是线性代数}
所以,任意两个实数$y_0, z_0\in \mathbb R$ 都决定了一组解。解空间是 ``\alert{二维}''的。不正式地说,给定四个\alert{变量},两个\alert{方程},解空间的\alert{自由度}是 $4 - 2 = 2.$

\vspace{12px}

现在来回答这个问题——什么是线性代数?

\begin{itemize}
    \item (入门级)线性代数就是把上面这句话\alert{严格化}的过程。
\end{itemize}

\end{frame}

\begin{frame}
\frametitle{为什么要严格化}
例如
\[
\systeme*{w + x + y - z = 0 , x - 2y + 3z = 0, w + 2x - y + 2z = 0}
\]
虽然有三个方程,但是因为\alert{第三个是前两个之和},所以解空间的维数还是$2$. 

怎么从概念上把这件事情说清楚呢?

\begin{itemize}
    \item (悄悄地告诉学过的人)系数矩阵的秩 (rank) 等于$2$。
\end{itemize}

\end{frame}

\begin{frame}
\frametitle{方程组的向量形式和矩阵形式}
最初那个方程,可以写成\alert{向量形式}
\[
w\begin{pmatrix} 1 \\ 0 \end{pmatrix} + x \begin{pmatrix} 1  \\ 1 \end{pmatrix}+ y\begin{pmatrix} 1 \\ -2 \end{pmatrix} + z\begin{pmatrix} -1 \\ 3 \end{pmatrix} = \begin{pmatrix} 0 \\ 0 \end{pmatrix}
\]
——哪些系数使得平面上的这四个向量的\alert{线性组合}是零向量?

\end{frame}

\begin{frame}
\frametitle{线性相关/线性无关}
给定一组向量$\{v_1, v_2, ... , v_n\}\subset \mathbb R^m$, 如果$a_1v_1 + \cdots + a_n v_n = \vec{0}$ 有\alert{非零解} $(a_1, a_2, \ldots, a_n)$, 则说这组向量是线性相关的。

\begin{itemize}
	\item 上面的四个 $\mathbb R^2$ 中的向量是线性相关的。
	\item $\begin{pmatrix} 1 \\ 0 \end{pmatrix}, \begin{pmatrix} 1  \\ 1\end{pmatrix}$ 是线性无关的。
\end{itemize}
\end{frame}

\begin{frame}

\frametitle{方程组的向量形式和矩阵形式}
最初那个方程,也可以写成\alert{矩阵形式}
\[
\begin{pmatrix} 1 & 1 & 1 & -1 \\ 0 & 1 & -2 & 3 \end{pmatrix} \begin{pmatrix} w \\ x \\ y \\ z \end{pmatrix} = \begin{pmatrix} 0 \\ 0 \end{pmatrix}
\]
——$\mathbb{R}^4$中的哪些向量在线性变换
\[T: \vec{v}\mapsto A\vec{v}\] 
下被映射到$\vec{0}\in\mathbb{R}^2$?($A$ 是上面的矩阵。)

\end{frame}

\begin{frame}
\frametitle{$\mathbb R ^n$ 到 $\mathbb R ^m$ 的线性映射}

对于映射(多元向量值函数)$T:\mathbb R ^n\to \mathbb R ^m$,如果 $T$ 满足 $T(a\vec{u}+b\vec{v}) = aT(\vec{u})+bT(\vec{v})$, 则称 $T$ 是 $\mathbb R ^n$ 到 $ \mathbb R ^m$ 的\alert{线性映射}。

\begin{itemize}
    \item 任意一个$\mathbb R ^n$ 到 $\mathbb R ^m$ 的线性映射都可以写成$\vec{v}\mapsto A\vec{v}$的形式。
    \item $T\begin{pmatrix} x \\ y \end{pmatrix} = 3x + 2y$ 是$\mathbb R ^2$ 到 $\mathbb R ^1$ 的线性映射。
    \item $\sin : \mathbb R \to \mathbb R$ \alert{不是}线性映射。因为$\sin(x+y)\neq \sin x + \sin y$.
    \item $f(x) = x^2 : \mathbb R \to \mathbb R$ \alert{不是}线性映射。因为$(x+y)^2\neq x^2 + y^2$.
    \item 这是这门课里\alert{最重要的概念}之一,多花点时间。
\end{itemize}


\end{frame}

\begin{frame}
\frametitle{不可轻视的``重写''}
虽然几种方程组写法是\alert{等价}的。但是却能带来概念上的飞跃。

\begin{itemize}
    \item 最初的写法(传统的线性方程组)更关心方程的\alert{单个解}。
    \item 向量的线性组合更多地从\alert{几何的角度}观察。
    \item 矩阵形式引入了一个\alert{函数(线性映射)},更加从整体的角度考虑。
    \item 不仅要解方程$A\vec{v} = \vec{0}$, 还要对任意的$\vec{b}$解方程$A\vec{v} = \vec{b}$.
\end{itemize}

\end{frame}

\begin{frame}
	有问题可以提问,提问完讲\alert{高斯消元法}。
\end{frame}

\begin{frame}
\frametitle{高斯消元法}
	高斯消元法只不过是把我们熟悉的``考虑方程$(1)$减去$3\times$方程$(2)$''的过程\alert{系统化}而已。
	但这很可能是很多人接触到的第一个\alert{算法}(竖式乘法不算的话),所以我们详细讲讲。

考虑方程

\[
A\vec{v} = \begin{pmatrix} 1 & 1 & 1 & -1 & 1 \\ 2 & 3 & -2 & 3 & 1 \\ 0 & 3 & -2 & 3 & 1 \end{pmatrix} \begin{pmatrix} u \\ w \\ x \\ y \\ z \end{pmatrix} = \begin{pmatrix} 0 \\ 0 \\ 0 \end{pmatrix}
\]

如何\alert{描述一个程序},使得计算机也可以解这个方程组?

\end{frame}

\begin{frame}
\frametitle{``算法''——行变换}
\[
\begin{pmatrix} 
1 & 1 & 1 & -1 & 1 \\ 
2 & 3 & 2 & 1 & 2 \\ 
0 & 3 & 1 & 8 & 1 
\end{pmatrix} \rightarrow
\begin{pmatrix} 
1 & 1 & 1 & -1 & 1 \\ 
0 & 1 & 0 & 3 & 0 \\ 
0 & 3 & 1 & 8 & 1 
\end{pmatrix} \rightarrow
\begin{pmatrix} 
1 & 1 & 1 & -1 & 1 \\ 
0 & 1 & 0 & 3 & 0 \\ 
0 & 0 & 1 & -1 & 1 
\end{pmatrix} 
\]
\[
\rightarrow
\begin{pmatrix} 
1 & 0 & 1 & -4 & 1 \\ 
0 & 1 & 0 & 3 & 0 \\ 
0 & 0 & 1 & -1 & 1 
\end{pmatrix} 
\rightarrow
\begin{pmatrix} 
1 & 0 & 0 & -3 & 0 \\ 
0 & 1 & 0 & 3 & 0 \\ 
0 & 0 & 1 & -1 & 1 
\end{pmatrix} 
\]
\end{frame}

\begin{frame}
\frametitle{包治百病的高斯消元法}
高斯消元法,也就是初等的行变换,能解决的问题很多。包括但不限于
\begin{itemize}
	\item 解线性方程组
	\item 判断线性相关/线性无关
	\item 矩阵求逆($(A,I)$经过初等行变换可以变成$(I, A^{-1})$)
	\item 求矩阵的秩(最后的矩阵里首项的$1$的个数)
	\item 计算行列式(用到高斯消元法的一个变体)
	\item 找子空间的一组基
	\item ......
\end{itemize}

\end{frame}

\begin{frame}
\frametitle{从``算法''的结果读取答案}
回到上面的问题,最后一个矩阵对应的方程组是
$
\systeme*{u = 3y , w = -3y, x = y - z}
$

所以$y, z$ 是\alert{自由变量},任意一个解都可以写成
\[
\begin{pmatrix} u \\ w \\ x \\ y \\ z \end{pmatrix} =  y_0\begin{pmatrix} 3 \\ -3 \\ 1 \\ 1 \\ 0 \end{pmatrix}+ z_0\begin{pmatrix} 0 \\ 0 \\ -1 \\ 0 \\ 1 \end{pmatrix}
\]
	
\end{frame}

\begin{frame}
\frametitle{对解集的描述}
进一步考察上面的例子。上述方程组的\alert{解集}是$\mathbb R^5$ 的一个子集。

这是 $\mathbb R^5$ 的一个二维\alert{子空间}(线性空间的概念待会会进一步讲)。

有两个向量$\begin{pmatrix} 3 \\ -3 \\ 1 \\ 1 \\ 0 \end{pmatrix}, \begin{pmatrix} 0 \\ 0 \\ -1 \\ 0 \\ 1 \end{pmatrix}$ 使得任意一个解都\alert{可以唯一地表示}成上述两个向量的线性组合。

这样的两个向量叫做这个子空间的一组\alert{基}(basis)。基里向量的个数叫做这个子空间的\alert{维数}。

\end{frame}

\begin{frame}
\frametitle{基的意义}
$k$维子空间的一组\alert{基}给定了把子空间和$\mathbb R^k$ \alert{等同}的一种方式。

把$\begin{pmatrix} 3 \\ -3 \\ 1 \\ 1 \\ 0 \end{pmatrix}, \begin{pmatrix} 0 \\ 0 \\ -1 \\ 0 \\ 1 \end{pmatrix}$ 取作上述子空间的一组基。在\alert{取定了这组基}的情况下,$\begin{pmatrix} u \\ w \\ x \\ y \\ z \end{pmatrix}$ 包含的信息,和$\begin{pmatrix} y_0\\z_0 \end{pmatrix}$ 一样多。
\end{frame}

\begin{frame}
\frametitle{基的意义:子空间里的坐标}
也就是说,由于这个子空间是个二维的(平面),我们可以\alert{选一个坐标系},使得里面每个点,都有唯一的坐标$\begin{pmatrix} y_0\\z_0 \end{pmatrix}$.

例子:

$\begin{pmatrix} 3 \\ -3 \\ 0 \\ 1 \\ 1\end{pmatrix}$ 的坐标就是 $\begin{pmatrix} 1\\1 \end{pmatrix}$

注意,\alert{对于不同的基,坐标不一样}。
\end{frame}

\begin{frame}
\frametitle{$\mathbb R^n$ 的子空间}

定义:对 $\mathbb R^n$ 的子集 $V \subset \mathbb R^n$, 如果对于任意的 $u, v\in V, a, b\in \mathbb R$, 都有 $au+bv\in V$, 则说 $V$ 是 $\mathbb R^n$ 的子空间。
\vspace{12px}

一句话:``子空间就是\alert{对线性组合封闭}的子集''。

(任意两个子空间中的向量线性组合之后还在子空间中)
\vspace{12px}

也可以分成两点验证:
\begin{itemize}
    \item 如果对于任意的 $u, v\in V$, 都有 $u+v\in V$.
    \item 如果对于任意的 $v\in V, c\in \mathbb R$, 都有 $cv\in V$.
\end{itemize}
就说 $V$ 是 $\mathbb R^n$ 的子空间。

\end{frame}

\begin{frame}
\frametitle{$\mathbb R^n$ 的子空间的具体形态}

对于 $m \times n$ 矩阵 $A$,考虑由$\vec{v}\mapsto A\vec{v}$ 定义的线性映射 $T:\mathbb R ^n\to \mathbb R ^m$,则
\begin{itemize}
    \item (零空间是子空间)$\{\vec{v}\in\mathbb{R}^n|A\vec{v} = \vec{0}\}$ 是 $\mathbb R ^n$ 的子空间。
    \item (像是子空间)$\{\vec{b}\in\mathbb{R}^m|A\vec{v} = \vec{b}\text{有解}\}$ 是 $\mathbb R ^m$ 的子空间。
\end{itemize}

要描述$\mathbb R^n$的一个子空间,找一个线性映射把它描述成其零空间或者像。
\end{frame}

\begin{frame}
\frametitle{为什么要研究 $\mathbb R^n$ 的子空间?}
因为子空间极大地简化了问题的复杂性。前面我们考虑了$\mathbb R^5$ 的一个二维子空间。$\mathbb R^5$ 中的一个向量要用$5$个数字描述,但是,如果我们知道它是那个子空间中的向量,就只需要用$2$个数字来描述,所以问题变简单了。

\vspace{12px}

建议大家不要用``$n$维向量''这种说法。维数永远是用来描述线性空间或者子空间的。

\vspace{12px}

如果从 $5$ 到 $2$ 的简化让人印象不够深刻的话,可以说个实际的用途,比如机器学习最简单的例子,识别手写的数字,每张图片($28\text{px}\times 28\text{px}$)都是 $28\times 28 = 784$ 维线性空间中的一个向量……
\end{frame}

\begin{frame}
\frametitle{应用:说说机器学习吧}
只讲最基本的例子,即 MNIST 的手写数字分类问题。
\vspace{12px}
\includegraphics[height=6cm]{mnist.jpeg}
\end{frame}

\begin{frame}
\frametitle{三句话解释神经网络}
这个问题是这样解决的

\begin{itemize}
    \item 每张图片($28\text{px}\times 28\text{px}$)都是 $\mathbb{R}^{28\times 28 = 784}$ 中的一个向量

    \item 希望找一个映射$T$把它映射到 $\mathbb R^{10}$, 使得

\begin{itemize}
    \item  如果一个图片$v$上的数字是$1$, 则$Tv = (1,0,0,..., 0)$ 
    \item  如果一个图片$v$上的数字是$2$, 则$Tv = (0,1,0,..., 0)$ 
    \item  ...... 
    \item  如果一个图片$v$上的数字是$0$, 则$Tv = (0,0,0,..., 1)$ 
\end{itemize}

	\item 能找到,问题就解决了。
\end{itemize}

\end{frame}

\begin{frame}
\frametitle{三句话解释神经网络}

对 $\mathbb{R}^{n}$ 里的向量组成的数据集。
\begin{itemize}
	\item 用线性变换和一种固定的非线性变换来提取特征,变到另一个线性空间里。
	\item 继续用线性变换和那个固定的非线性变换来提取特征,再变到另一个线性空间里。
	\item ......
	\item 直到最后落在 $\mathbb R^{10}$ 中。

\end{itemize}
\end{frame}

\begin{frame}
\frametitle{三句话解释神经网络}

\begin{itemize}
	\item 根据误差反馈来的信息来修正我们假设的线性变换,以减小误差。
	\item ——修正的依据是多元微积分里的链式法则。
	\vspace{12px}
	\item 误差较小的时候能得到比较精确的结果。
	\item \alert{理论上}只用到一点多元微积分和线性代数——不超过大学二年级的数学。

\end{itemize}
\end{frame}

\begin{frame}
\frametitle{谢谢大家}
自由提问时间
\end{frame}



\end{document}